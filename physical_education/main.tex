\documentclass[bachelor, och, referat ]{SCWorks}
\usepackage[T2A]{fontenc}
\usepackage[cp1251]{inputenc}
\usepackage[english,russian]{babel}
\newcommand{\eqdef}{\stackrel {\rm def}{=}}
\usepackage{indentfirst}
\newtheorem{lem}{Лемма}

\begin{document}

% Кафедра (в родительном падеже)
\chair{математической кибернетики и компьютерных наук}

% Тема работы
\title{Моя группа здоровья}

% Курс
\course{1}

% Группа
\group{111}

\napravlenie{02.03.02 "--- Фундаментальная информатика и информационные технологии}

\studenttitle{Студентки}


\author{Кляулиной Светланы Борисовны}

% Заведующий кафедрой
\chtitle{доцент, к.\,ф.-м.\,н.} % степень, звание
\chname{С.\,В.\,Миронов}

%Научный руководитель (для реферата преподаватель проверяющий работу)
\satitle{} %должность, степень, звание
\saname{В.\,Г.\,Беляева}

% Год выполнения отчета
\date{2022}

\maketitle

\section*{Введение.}
Жизнь человека зависит от состояния здоровья организма и масштабов использования его психофизиологического потенциала. Все стороны человеческой жизни в широком диапазоне социального бытия в конечном счёте, определяются уровнем здоровья.

Хорошее здоровье характерно основному условию для выполнения человеком его биологических и социальных функций, некий фундамент самореализации личности. Оно даёт хорошее самочувствие, способность успешно переносить большие нагрузки, уверенность в своих силах, более быстрое и полноценное течение процессов восстановления после умственной деятельности. Чтобы быть здоровым и выносливым - нужны собственные усилия постоянные и значительные.

Массовый спорт даёт возможность миллионам людей со¬вершенствовать свои физические качества и двигательные возможности, укреплять здоровье и продлевать творческое долголетие, а значит, противостоять нежелательным воздействиям на организм современного производства и условий повседневной жизни.

Задачи массового спорта во многом повторяют задачи физической культуры, но реализуются спортивной направленностью регулярных занятий и тренировок. К элементам массового спорта значительная часть молодёжи при¬общается ещё в школьные годы. Именно массовый спорт имеет наибольшее распространение в студенческих коллективах.

Студенты колледжей, вузов, университетов в зависимости от физического развития, состояния здоровья и функциональной подготовки разделены на 3 группы: основную, подготовительную и специальную. 
\newpage

\section{1. Подготовительная группа здоровья.}
К подготовительной медицинской группе относятся практически здоровые студенты, имеющие некоторые морфофункциональные отклонения либо слабо подготовленные физически; входящие в группы риска по возникновению патологии или с хроническими заболеваниями в стадии стойкой ремиссии не менее 3-5 лет.

Обучающимся, отнесённым к этой группе, разрешается заниматься физической культурой по учебной программе при условии постепенного освоения комплекса двигательных навыков и умений, связанных с предъявлением к организму повышенных требований, более строгой дозировки физической нагрузки и исключения противопоказанных движений.
\newpage

\section{2. Контрольные нормативы.}
Нормативы подготовительной группы по физкультуре несколько ниже, нежели для прочих.
\begin{enumerate}
\item{Поднимание и опускание туловища из положения лежа, ноги закреплены, руки за головой.}
\item{Тест на силовую подготовленность – сгибание-разгибание рук в упоре лежа.}
\item{Измерение гибкости.}
\item{Прыжок в длину с места.}
\end{enumerate}
Стоит отметить, что причисление к группе, в которой применяются сниженные нормативы, обусловлено не противопоказаниями к спортивной активности, а лишь возможной опасностью при повышенных нагрузках. В то же время определённые регулярные тренировки необходимы для полноценного развития детского организма. 
\newpage

\section{3. Основные требования безопасности на месте занятий физической культурой.}
Во время занятий физической культурой следует соблюдать правила техники безопасности. Большое значение имеет подготовка мест занятий, наличие подготовленного исправно спортивного оборудования и инвентаря. 
\begin{itemize}
\begin{itemize}
\itemЗаниматься только с учителем или его помощником, обязательно выполнять все их требования (запрещается без педагога выполнять сложные и неизвестные упражнения).

\itemОбязательное выполнение разминки, способствующей разогреванию основных групп мышц.

\itemНе покидать без разрешения учителя место занятий.

\itemВыполнение упражнений только на исправном оборудовании; бережно относиться к инвентарю и оборудованию; закончив выполнение упражнений с инвентарём (мячи, палки, скакалки…) класть его в место его хранения (специально отведённое место).

\itemСоответствие спортивной одежды и обуви занятиям физической культурой, погодным и другим условиям. Обувь должна быть чистой и с нескользкой подошвой.

\itemОдежду и обувь для физической культуры необходимо приносить с собой в сумке (пакете). Перед тем, как приступить к занятиям необходимо, надеть спортивный костюм и обувь. После окончания занятий необходимо снять спортивный костюм и обувь, надеть школьную форму (или другую одежду и обувь), вымыть лицо и руки с мылом.

\itemНа спортивной площадке и в зале не сорить, следить за чистотой.

\itemУчастие в занятиях только при хорошем самочувствии.

\itemВо время бега на короткие дистанции нельзя перебегать на соседнюю дорожку, это может привести к столкновение учащихся. Все беговые соревнования проводят при движения в одном направлении.
\end{itemize}
\end{itemize}

\newpage

\section{4. Абсолютные противопоказания к занятиям физической культурой.}
Следующие пункты являются абсолютными противопоказаниями:
\begin{itemize}
\begin{itemize}
\itemнедостаточность кровообращения II—III степени;
\itemострый инфаркт миокарда;
\itemактивная фаза ревматизма, миокардит; 
\itemстенокардия покоя; 
\itemэмболия лёгочной артерии; 
\itemинфаркт трансмуральный; 
\itemаневризма аорты; 
\itemострое инфекционное заболевание; 
\itemтромбофлебит и сердечная недостаточность; 
\itemтахикардия покоя, экстрасистолия и другие нарушения ритма; 
\itemстеноз аорты и почечной артерии; 
\itemмиопия (близорукость) более 7 диоптрий.
\end{itemize}
\end{itemize}
\newpage

\section*{Заключение.}
Характеристики групп здоровья для студентов описаны в приказе Министерства здравоохранения. К основным критериям оценки состояния организма относят наличие или отсутствие хронического заболевания, работоспособность систем, степень сопротивляемости организма к внешней среде, уровень физического развития.
Можно сказать, что учебный процесс при деление на медицинские отделения имеет определённую специфику и преимущественно направлен на:
\begin{itemize}
\begin{itemize}
\itemукрепление здоровья студентов, устранение функциональ¬ных отклонений, недостатков в физическом развитии и физичес¬кой подготовленности на протяжении всего периода обучения;
\itemиспользование студентами знаний о характере и течении своего заболевания, самостоятельное составление и выполнение комплексов общеразвивающих и специальных упражнений, на¬правленных на профилактику болезни;
\itemприобретение студентами необходимых знаний по основам теории, методики и организации физического воспитания.
\end{itemize}
\end{itemize}
Умение контролировать своё физическую и психоэмоциональную состояние важно. 
Современные условия жизни предъявляют повышенные требования к здоровью и интеллектуальным возможностям человека. Проблема сохранения здоровья занимает важное место в системе социальных ценностей и приоритетов общества. А умение контролировать эту проблему очень важно, поэтому и  распределение человек в верное медицинское отделение только приумножит его успехи в этом.
\newpage

\section*{Литература.}
\begin{enumerate}
\item{https://www.informio.ru/publications/id2959/Osobennosti-provedenija-zanjatii-so-studentami-podgotovitelnoi-i-specialnoi-medicinskih-grupp}
\item{https://studopedia.su/15_100870_kak-formiruyutsya-gruppi-dlya-zanyatiy-fk-v-vuze.html}
\item{https://studfile.net/preview/5249723/page:9/}
\item{https://megaobuchalka.ru/1/7714.html}
\item{https://med-tutorial.ru/m-lib/b/book/4081953680/104}
\end{enumerate}
\end{document}