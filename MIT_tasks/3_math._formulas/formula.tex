\documentclass{article}
\usepackage[T2A]{fontenc}
\usepackage[utf8]{inputenc}
\usepackage{amsthm}
\usepackage{amsmath}
\usepackage{amssymb}
\usepackage{amsfonts}
\usepackage{mathrsfs}
\usepackage[12pt]{extsizes}
\usepackage{fancyvrb}
\usepackage{indentfirst}
\usepackage[
  left=2cm, right=2cm, top=2cm, bottom=2cm, headsep=0.2cm, footskip=0.6cm, bindingoffset=0cm
]{geometry}
\usepackage[english,russian]{babel}


\begin{document}
    \section*{Вариант 10}
    Решение уравнений с учетом краевых условий представим в виде
    
    \begin{equation}
    W=\sum_{k=1}^{\infty}\left[\left(R_{k}(\tau)+R_{k}^{0}\right) 
    \cos \frac{(2 k-1) \pi \xi}{2}+Q_{k}(\tau) \sin k \pi \xi\right].
    \label{eq:name}
    \end{equation}
    
    \noindentВерхний индекс 0 в (\ref{eq:name}) означает решение,
    соответствующее постоянному уровню давления
    $p_{0}$, независящему от $\tau$ .
    
    Принимая во внимание линейность предыдущего уравнения и
    подставляя в него (\ref{eq:name}), найденное выражение
    для давления, а, также раскладывая оставшиеся члены, входящие в его
    правую часть в ряды по тригонометрическим функциям, 
    из полученного уравнения запишем
    выражение для составляющей, независящей от времени $R_{k}^{0}$
    \begin{equation*}
    R_{k}^{0}=(2 \ell /((2 k-1) \pi))^{4}\left(4(-1)^{k+1} 
    /((2 k-1) \pi)\right) p_{0}\left(D w_{m}\right)^{-1},
    \end{equation*}
    и уравнения для определения $R_{k}(\tau)$  и  $Q_{k}(\tau)$
    \begin{equation}
    a_{1 c k} w_{m} R_{k}+a_{2 c k} w_{m} d R_{k} / d \tau=2(-1)^{k+1}
    /((2 k-1) \pi D) p_{m}^{+} f_{p}(\tau),
    \end{equation}
    \begin{equation}
    a_{1 s k} w_{m} Q_{k}+a_{2 s k} w_{m} d Q_{k} / d \tau=(-1)^{k+1}
    /(k \pi D) p_{m}^{+} f_{p}(\tau).
    \end{equation}

\end{document}

